\chapter{Introduction}
\label{chap:\currfilebase}

\section{Motivation}
\label{sec:motivation}

\section{Outline}
\label{sec:outline}

\section{State of the Art}
\label{sec:state_of_art}

% The demand for increasing geometric accuracy of precision 5-axis machine tools has reduced the allowed error motions to the extent that error reduction does not suffice while maintaining the productivity and repeatability in a production facility environment. A number of error compensation methods established themselves when it commes to dynamic error motions. Although one of the main reasons causing inaccuracy of machine tools are thermal error motions~{\cite{mayr2012thermal}} the tools used to compensate them specifically are few and far between.\par

% %-----------------------------------------------------
% \section{Target}
% The target of this thesis is to define and carry out measurement tests that allow the characterization of thermally induced error motions on a linear axis of a 5-axis machine tool. The results derived of these tests shall be analysed for a correlation between the temperatures and the deviations measured as well as implemented in a phenomenological model that acts as a non-physical input-output model.

% %-----------------------------------------------------
% \section{Outline}
% First previous studies and the field of research are introduced in the state of the art. Followed by identifying different measurement set-ups and measurement procedures to characterize the thermal error motions occurring along a linear axis in Chapter~{\ref{chap:measurements}}. The evaluation software used to filter and plot the data gathered during the measurement tests is described in the next chapter and the illustration and discussion of the measurement results may be found in Chapter~{\ref{chap:measurementresults}}.  Furthermore a first order model is proposed in Chapter~{\ref{chap:phenomenologicalmodel}}, which represents an area of application the measurement results may be used in future needs. Lastly, a final look on the main difficulties during this thesis and the utility of its results are written down in Chapter~{\ref{chap:conclusionandoutlook}}.

% %-----------------------------------------------------
% \chapter{State of the Art}

% The discussion by Mayr~{\cite{mayr2012thermal}} provides a brief overview of thermal errors in MTs and the known tools to reduce them. This thesis will address thermal geometric errors of a linear axis~{\cite{iso230_3_eng2007}}, their measurement~{\cite{iso230_1_eng2012,iso10791_10_eng2007,iso13041_8_eng2004}} and derived from prior, the modelling of the thermally induced geometric error of the tool center point to the workpiece. Considering the investigations of a similar issue on rotational axes on the very same 5-axis machine tool~{\cite{gebhardt2014high}}, we're given a solid foundation that can be implemented to that task.\par
% The test algorithms are based on a defined standard test~{\cite{iso230_2_eng2014}}. An overview of measurement methods for geometric error motions is provided by Schwenke~{\cite{schwenke2008geometric}} and measurement set-ups espacially suited for 5-axis machine tools are listed by Ibaraki~{\cite{ibaraki2013indirect}}. The utility of a phenomenological model in this area of reasearch is described by Blaser~{\cite{blaser2017adaptive}}.
% In addition the temperature distribution of a three axis milling machine tool was analysed by Huang~{\cite{huang2015real}} and the way of implementing a thermal compensation model to a CNC is described by Liu~{\cite{liu2017robust}}.

% \section{Thermal Sources}
% A multitude of heat sources can cause thermal error motions~{\cite{blaser2017adaptive}}. Considering one linear axis, the main heat sources reduce to environmental temperature change and the heat induced by moving components of the regarded axis. Shi~{\cite{shi2015investigation}} approaches this by modelling a ball screw feed drive expansion in a simplified thermo-mechanical system. In this system the heat is induced through the power loss of the electric motor, the friction between the balls and the races of the two bearings and the friction between the balls and the grooves of the nut (Figure~{\ref{fig:shiballscrewheat}}). Additionally heat convection with the room temperature and the emissivity of all components of the model except the ball screw are neglected due to their significantly smaller surface.

% \begin{figure}[!ht]
% \centering
% \includegraphics[width=0.6\linewidth]{figures/introduction/shiballscrewheat}
% \caption{Thermal consideration of a ball screw drive~{
% \cite{shi2015investigation}}}
% \label{fig:shiballscrewheat}
% \end{figure}

% Note that apart of the heat sources mentioned, the coolant fluid adds an important factor to the thermal behaviour of the machine tool.

% \subsection*{Temperature Gradients}
% MTs are affected by thermal gradients in a variety of ways. Thermal gradients can namely lead to thermal motion errors and occur because of heat sources that exist within the boundaries of the environment. Their existens implies that different parts of the environment have different mean temperatures~{\cite{iso230_3_eng2007}}. But the temperature is not a function of location only. The work of Zhang~{\cite{zhang2017thermal}} shows the periodical fluctuations of the time-varying environmental temperature. And furthermore the investigations conducted by Mayr~{\cite{mayr2015simulation}} concluded in a relation between the environmental temperature change frequency and the TCP displacements occurring on a MT 
% (Figure~{\ref{fig:mayrtemperaturebode}}).

% \begin{figure}[!ht]
% \centering
% \includegraphics[width=0.8\linewidth]{figures/introduction/mayrtemperaturebode}
% \caption{Bode plot of TCP displacements at a measurement position (P1) on the considered X-axis for different environmental change frequencies with an amplitude of {\SI{1}{\kelvin}}. {\SI{100}{\percent}} is the maximum amplitude at axis position P1, $-\SI{210}{\mm}$~{\cite{mayr2015simulation}}}
% \label{fig:mayrtemperaturebode}
% \end{figure}

% \section{Error Motions of Linear Axes}
% The three unwanted transitional movements of a moving component, called linear error motions of a linear axis, are separated into one linear positioning error motion along the direction of motion and two straightness error motions perpendicular to the regarded axis.\par
% The three unwanted rotational movements of a moving component, called angular error motions of a linear axis, are defined as rotations around the three orthogonal axes. They can be separated into one roll motion (around the direction of motion) and two tilts. Note that if the considered axis is horizontal, the tilt around the vertical direction can be called yaw and the tilt around the horizontal direction perpendicular to the direction of motion can be called pitch. The positive sign of the angular error follows the right-hand rule. 

% %!!!(preferably y-axis)
% \begin{figure}[!ht]
% \centering
% \includegraphics[width=0.6\linewidth]{figures/introduction/errorMotions}
% \caption{Error motions of linear axes~{\cite{iso230_1_eng2012}}}
% \label{fig:errorMotions}
% \end{figure}

% where
% \begin{table}[!ht]
% \centering
% \begin{tabular*}{0.9\linewidth}{l p{0.75\linewidth}}
%   \textbf{Symbol} & \textbf{Description} \\ 
%   \midrule
%   $1$				& X-axis commanded linear motion \\
%   $\mathrm{EAX}$	& angular error motion around A-axis (roll) \\
%   $\mathrm{EBX}$	& angular error motion around B-Axis (yaw) \\
%   $\mathrm{ECX}$	& angular error motion around C-axis (pitch) \\
%   $\mathrm{EXX}$	& linear positioning error motion of X-axis; positioning deviations of X-axis \\
%   $\mathrm{EYX}$	& straightness error motion in Y-axis direction \\
%   $\mathrm{EZX}$	& straightness error motion in Z-axis direction \\
% \end{tabular*}
% \caption{Legend to Figure~{\ref{fig:errorMotions}}}
% \label{tab:errorMotions}
% \end{table}
% %!!!
% \section{Measurement of straightness error motions}

% \subsection*{Straightedge and linear displacement sensor}
% In this comparative measurement method a straightedge is used as reference for straightness. A displacement sensor mounted close to the functional point of the moving component allows the measurement of straightness deviations in horizontal or vertical direction depending on the setup of the straightedge.\par
% Unknown errors of the straightedge can be determined and removed from the straightness error motion measurement using the straightedge reversal method. This method applies in the horizontal plane only because of deflection due to gravity in a vertical setup.

% \begin{figure}[!ht]
% \centering
% \subfigure[Normal setup]{
%   \includegraphics[width=0.4\linewidth]{figures/introduction/straightedgenormal}
%   \label{fig:straightedgenormal}
% }\hspace{0.08\linewidth}
% \subfigure[Reversed setup]{
%   \includegraphics[width=0.36\linewidth]{figures/introduction/straightedgereversed}
%   \label{fig:straightedgereversed}
% }
% \caption{Straightedge and linear displacement sensor~{\cite{iso230_1_eng2012}}}
% \label{fig:straightedge}
% \end{figure}

% \begin{table}[!ht]
% \centering
% \begin{tabular*}{0.5\linewidth}{l p{0.4\linewidth}}
%   \textbf{Key} &\\ 
%   \midrule
%   $1$	& straightedge \\
%   $2$	& measurement line \\
%   $3$	& straighedge support points (3) both sides \\
%   $4$	& linear displacement sensor \\
%   $5$	& machine table \\
% \end{tabular*}
% \caption{Legend to Figure~{\ref{fig:straightedge}}}
% \label{tab:straightedge}
% \end{table}

% \subsection*{Microscope and taut wire}
% As straightness reference a steel wire with a diameter near to \SI{0.1}{\mm} is stretched to be approximately parallel to the direction of motion to be checked. The straightness errors are measured using a microscope, or other displacement sensors capable of registering the taut wire deviation perpendicular to the direction of motion. The sensor shall be mounted close to the functional point of the moving component (see Figure~{\ref{fig:microscopetaut}}). The straightness error motion results out of the deviation between taut wire and sensor.\par
% It is not recommended to measure straightness error motions in a vertical plane using this method because it is difficult to determine the sag at any given point.

% \begin{figure}[!ht]
% \centering
% \includegraphics[width=0.35\linewidth]{figures/introduction/microscopetaut}
% \caption{Staightness error measurement using taut wire and microscope~{\cite{iso230_1_eng2012}}}
% \label{fig:microscopetaut}
% \end{figure}

% \begin{table}[!ht]
% \centering
% \begin{tabular*}{0.3\linewidth}{l p{0.2\linewidth}}
%   \textbf{Key} &\\ 
%   \midrule
%   $1$	& spindle \\
%   $2$	& microscope \\
%   $3$	& taut wire \\
%   $4$	& weight \\
%   $5$	& table \\
% \end{tabular*}
% \caption{Legend to Figure~{\ref{fig:microscopetaut}}}
% \label{tab:microscopetaut}
% \end{table}

% \subsection*{Alignment telescope}
% When using an alignment telescope the telescope shall be mounted on the table and the target shall be mounted on the tool holder. The optical axis of the telescope serves as straightness reference (see Figure~{\ref{fig:alignmenttelescope}}).\par
% Local bending causes the optical line of the telescope to change position. Therefore one should take care in the fixing of the telescope, particularly in situations where bending is suspected.

% \begin{figure}[!ht]
% \centering
% \includegraphics[width=0.6\linewidth]{figures/introduction/alignmenttelescope}
% \caption{Straightness error measurement using alignment telescope~{\cite{iso230_1_eng2012}}}
% \label{fig:alignmenttelescope}
% \end{figure}

% \begin{table}[!ht]
% \centering
% \subtable{%
%   \centering
%   \begin{tabular*}{0.32\linewidth}{l p{0.22\linewidth}}
%     \textbf{Key} &\\ 
%     \midrule
%     $1$	& workpiece side (table) \\
%     $2$	& tool side (position 1) \\
%     $3$	& tool side (position 2) \\
%     $4$	& telescope \\
%     $5$	& reading micrometer \\
%   \end{tabular*}
% }\hspace{0.08\linewidth}
% \subtable{%
%   \centering
%   \begin{tabular*}{0.32\linewidth}{l p{0.22\linewidth}}
%     \textbf{Key} &\\ 
%     \midrule
%     $6$	& reticule \\
%     $7$	& target \\
%     $8$	& light source \\
%     $9$	& measured deviation \\
%   \end{tabular*}
% }%
% \caption{Legend to Figure~{\ref{fig:alignmenttelescope}}}
% \label{tab:alignmenttelescope}
% \end{table}

% \subsection*{Alignment laser}
% Similarly to the alignment telescope the laser head shall be mounted on the component that carries the workpiece and the four-quadrant photo-diode target shall be mounted on the tool carrying side.\par
% The problem of local bending can best be stemmed by fixing the alignment laser on a support simulating a rigid workpiece connected to the table.

% \subsection*{Laser straightness interferometer}
% To measure the relative straightness error motion between the tool and the workpiece, the bi-mirror reflector shall be mounted on the component that carries the workpiece and the Wollaston prism shall be mounted on the component that carries the tool. Optical components and measuring methods differ and should be applied following the manufacturers' instructions.\par
% Local bending causes the centreline of the reflector to change its position. This situation can be rectified by supporting the reflector mounting kinematically.

% %!!! figure?!

% \section{Measurement of linear positioning error motions}

% \subsection*{Laser interferometer\label{sec:laserinterf}}
% The retroreflector mounted on the component that carries the tool and the interferometer mounted on the table allow to measure the relative positioning error motion between the tool and the workpiece. The laser beam emitted from a leaser head shall be parallel to the linear motion as much as possible as misaligning causes cosine error. To compensate for air refraction air sensors, measuring air temperature, pressure and humidity, shall be located near to the beam path.

% %!!! figure?!

% \subsection*{Linear encoder}
% To measure the position between workpiece and tool a scale and a reader shall be mounted on the workpiece and the tool carrying component respectively. The scale shall be mounted as parallel as possible to the measured axis of motion as misalignment causes cosine error.\par
% It is possible to measure one straightness error motion simultaneously when a grid is used as scale.

% %!!! figure?!
% \subsection*{Calibrated ball array}
% The positions of the precision spheres of the ball array artefact (see Figure~{\ref{fig:ballarar}}) in the machine coordinate system are determined by a displacement measuring or a surface detection system. The calibration documentation typically includes the center position of the individual spheres, the sphere size and form measurement uncertainty and the artefact coefficient of thermal expansion. The calibrated center points are normally not exactly equally spaced and thus the position of the target points prescribed in Section~{\ref{sec:meastestalgo}} is partially fulfilled.

% \begin{figure}[!ht]
% \centering
% \subfigure[1D ball array]{
%   \includegraphics[width=0.22\linewidth]{figures/introduction/ballarray1d}
% }\hspace{0.08\linewidth}
% \subfigure[2D-ball array]{
%   \includegraphics[width=0.35\linewidth]{figures/introduction/ballarray2d}
% }
% \caption{Ball array artefacts}
% \label{fig:ballarar}
% \end{figure}

% A 2D-ball artefact may be used for a 3D-ball plate measurement~{\cite{bringmann2009machine,liebrich2009calibration}}.
% \section{Measurement of environmentally induced uncertainties}

% \subsection*{EVE test}
% The EVE test aims to reveal effects of environmental changes, namely the change in environmental temperature. Its goal is to estimate the environmental error induced during other performance measurements~{\cite{iso230_3_eng2007}}. For this test, the fixture in which the linear displacement sensors are mounted shall be securely fixed to the table. Using five displacement sensors one is able to measure the displacements between the tool carrying component and the table as well as the tilt or rotation around the axes perpendicular to the spindle (Figure~{\ref{fig:eveTest}}). Simultaneously the temperature of the machine structure at a point of interest and the ambient temperature are to be measured. The point of interest mentioned is located as close to the spindle bearing or at a position agreed upon between the supplier/manufacturer and the user. The ambient temperature shall be represented by the air temperature outside the machine working space at a location where no warm-up effects of the MT itself have no impact on the temperature profile.

% \begin{figure}[!ht]
% \centering
% \includegraphics[width=0.8\linewidth]{figures/introduction/etveTest}
% \caption{Typicl set-up for testing EVE and thermal distortion of structure caused by rotating spindle and by moving linear axis~{\cite{iso230_3_eng2007}}}
% \label{fig:eveTest}
% \end{figure}

% \begin{table}
% \centering
% \begin{tabular*}{0.5\linewidth}{l p{0.4\linewidth}}
%   \textbf{Key}	& \textbf{Description} \\
%   \midrule
%   1	& ambient air temperature sensor \\
%   2	& spindle bearing temperature sensor \\
%   3	& test mandrel \\
%   4	& linear displacement sensors \\
%   5	& fixture \\
%   6	& fixture bolted to table \\
% \end{tabular*}
% \caption{Legend to Figure~{\ref{fig:eveTest}}}
% \label{tab:eveTest}
% \end{table}

% \section{Measurement test algorithm\label{sec:meastestalgo}}
% ISO 230-2(E)2014~{\cite{iso230_2_eng2014}} suggests measurement tests for numerically controlled axes. Those tests focus on the determination of accuracy and repeatability of specific axes individually. Thus when considering a linear axis the positioning error motion is the main deviation to be tested.

% \subsection*{Selection of target positions}
% The target positions are to be approached by the MT. When a target point is reached, a position measurement shall be conducted to derive the deviation from the difference between the measurement position and the target point inserted into the machine program. Defining the target position is dependant on the range and the resolution of the measurements. Where the value of each target position can be chosen freely, one shall take the general form of Formula~{\ref{eqn:targetpos}}. Target positions selected for the execution of acceptance or reverification tests shall be different from the sampling points used for numerical compensation of the relevant axis positioning errors.
% \begin{align}
%   P_i &= (i-1)p+r \label{eqn:targetpos}
% \end{align}
% where the parameters are described as:

% \begin{table}[H]
% \centering
% \begin{tabular*}{0.9\linewidth}{l p{0.75\linewidth}}
%   $i$	& number of the current target position \\
%   $p$	& nominal interval based on a uniform spacing of target points over the measurement travel \\
%   $r$	& random number within $\pm$ one period of expected periodic positioning error (such as errors caused by the pitch variations of the ball screw and pitch variations of linear or rotary scales), used to ensure that these periodic errors are adequately sampled, and where, if no information on possible periodic errors is available, $r$ shall be within $\pm\SI{30}{\percent}$ of $p$ \\  
% \end{tabular*}
% \end{table}

% \subsection*{Test for linear axes up to 2000 mm}
% The measurements shall be made at all the target positions according to the standard test (Figure~{\ref{fig:standardtest}}). Where the TCP to table transition along the axis considered is alternated in its direction and interrupted at every target position for a defined duration. During this interruption the position measurement is to be conducted. The difference between the position measured and the target position will then deliver the deviation.

% \begin{figure}[!ht]
% \centering
% \includegraphics[width=0.5\linewidth]{figures/introduction/standardtest}
% \caption{Standard test~{\cite{iso230_2_eng2014}} --- Note that this representation only refers to the sequence of the target point to target point transition only.}
% \label{fig:standardtest}
% \end{figure}

% \begin{table}[!ht]
% \centering
% \begin{tabular*}{0.3\linewidth}{l p{0.2\linewidth}}
%   $a$	& Position $i$ ($m=8$) \\
%   $b$	& Cycle $j$ ($n=5$) \\
%   $c$	& Target points \\  
% \end{tabular*}
% \caption{Legend to Figure~{\ref{fig:standardtest}}}
% \label{tab:standardtest}
% \end{table}

% For evaluation the deviations at each target position in a test are firstly considered in the approach direction respectively. Where one shall define the boundaries of the deviation distribution according to:

% \begin{align}
% 	\bar{x}_i\uparrow\pm 2\cdot s_i\uparrow	&&\text{and}	&&\bar{x}_i\downarrow\pm 2\cdot s_i\downarrow
% \end{align}

% where $\bar{x}_i\uparrow$ and $\bar{x}_i\downarrow$ are the mean unidirectional positioning deviation at a position $P_i$ of all deviations $x_{ij}\uparrow$ or $x_{ij}\downarrow$ respectively of the cycles $j$ in a test over $n$ cycles:

% \begin{align}
% 	\bar{x}_i\uparrow	&= \frac{1}{n}\sum\limits_{j=1}^n x_{ij}\uparrow	&\text{and}	&&\bar{x}_i\downarrow	&= \frac{1}{n}\sum\limits_{j=1}^n x_{ij}\downarrow
% \end{align}

% The estimators for the unidirectional axis positioning repeatability at a position $s_i\uparrow$ and $s_i\downarrow$ are defined as follows:

% \begin{align}
% 	s_i\uparrow	&= \sqrt{\frac{1}{n-1}\sum_{j=1}^n\big( x_{ij}\uparrow-\bar{x}_i\uparrow\big)^2}	&\text{and}	&&s_i\downarrow	&= \sqrt{\frac{1}{n-1}\sum_{j=1}^n\big( x_{ij}\downarrow-\bar{x}_i\downarrow\big)^2}
% \end{align}

% Given the two mean unidirectional positioning deviations the mean bi-directional positioning deviations $\bar{x}_i$ at target position $i$ may be computed as the mean average.

% \begin{align}
% 	\bar{x}_i &= \frac{\bar{x}_i\uparrow+\bar{x}_i\downarrow}{2}
% \end{align}