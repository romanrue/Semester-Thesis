\chapter{Introduction}
\label{chap:\currfilebase}

For more than half a century the application of reinforced carbon fibre polymers has been pushing the design of lightweight structures to new limits. Over this time a number of macroscopic failure criteria evolved and the quality of products got more consistent through refined manufacturing processes \cite{gurit2017guide} and well established testing methods \cite{calsson2014experimental}.

In said failure criteria one aims to fit the failure curve to a small number of characteristic values. This common approach does not include intermediate values for the models and thus the measurement of such has been widely neglected \cite{daniel2016yield}.

To evaluate the established failure criteria with new materials and to compare them to newer criteria \cite{daniel2007failure,daniel2016yield} it is important to conduct experiments that do not just confirm the characteristic values, but that fill the gaps in between those values and estimate the overall fit quality of each criterion.

\section{Motivation}
\label{sec:motivation}

Taking advantage of a reliable automated testing system, a large database of tensile experiment has been created using off-axis unidirectional lamina specimen. This specimen uses the material anisotropy to create different stress states inside the lamina by changing the angle between the fiber and the tensile direction. The results of these experiments show a poor fitting between experimental data and theoretical failure criteria at low off-axis angle, leading to possible premature failure of the composite structure. A complimentary set of compressive experiments may lead to a stronger incentive to stronger adapt the theoretical failure criteria to the measurement data.

\section{Overview}
\label{sec:overview}

In this thesis the main focus lies in collecting data of a given experimental setup. All experiments are conducted on a test setup as described in \autoref{chap:test_setup}. Improvements and tweaks to the setup were possible due to the results in \autoref{chap:preliminary_testing}. Then the experimental results are shown and discussed in \autoref{chap:experimental_results}. Lastly the main takeaways of the thesis and the further use of the measurement data is highlighted in \autoref{chap:conclusion_outlook}.