\chapter{Conclusion and outlook}
\label{chap:\currfilebase}

A total of 74 combined loading compression measurement tests until failure on unidirectional lamina specimen with off-axis angles ranging from $\SI{0}{\degree}$ to $\SI{90}{\degree}$ have been conducted. The measurement data includes the load and displacement as well as the local strain through DIC of the CCD footage.

\subsection*{Conclusions}
\begin{itemize}
    \item The experimental results correspond to the expected pattern. Apart from some outliers the fatigue strengths at each off-axis angle show small variance, indicating an overall good repeatability of the tests.
    \item Specimens with off-axis angles of $\SI{30}{\degree}$ or higher can be tested without the use of tabs.
    \item The clamping force in combined loading compression measurements has a significant impact on the measurement results.
    \item It has been shown that 3D-printed positioners for both aligning tabs to the specimen and specimens to the test fixture are an efficient way to increase the test repeatability.
    \item The Tsai-Hill and Tsai-Wu failure criteria fit the intermediate off-axis angles well and are generally more conservative than the test results.
\end{itemize}

\subsection*{Outlook}
If ones focus lies in expanding the failure criteria, one could combine the experimental test results of this thesis with those of tension tests to validate the results under a failure surface or investigate more stress based or even strain based criteria. On the other hand if the focus lies within gaining more accurate experimental data, one could validate the data of this thesis with a different test setup or quantify the influence of test parameters that were left untouched until now. Namely the clamp screw torque and strain rate dependent testing. Furthermore it is possible to optimize the procedure and the software for the measurement preparation, operation and evaluation.

